\documentclass[UTF8]{ctexart}
\title{第14周学习记录}
\author{Allan}
\date{\today}
\begin{document}
\maketitle
\tableofcontents
\section{12月7日}
\paragraph{ 其实全文都是从那个.md文件里面搬过来的,主要为了实践一下latex的用法}
\subsection{LaTex}
下载并安装了LaTex,学习了部分基本格式。
\subsection{线性代数}
复习并做了两套综合卷,迎接明天的期末考试。
\section{12月8日}
\subsection{线性代数}
学习并复习了线性代数

通过观看MIT线性代数复习并巩固了以下知识:

矩阵乘法(四种方法):常规、列方法、行方法、列乘行。

逆矩阵:行列式不为零(非奇异矩阵)才存在,行列式为零时(奇异矩阵)不存在逆矩阵。

LU分解:将一个矩阵分为上三角阵与下三角阵相乘。

矩阵转置,向量空间,列空间和零空间(AX=0时X的所有解的集合所形成的空间)。

今日考试把基础解系和通解概念弄反了(反思)
\subsection{python}
学习了NumPy模块(python数据分析手册):

(引用时一般import numpy as np便于使用)

了解了python中数据的存储方式:是用c语言编写的利用指针和结构体进行存储的。

优点:是动态的,灵活性较高。(甚至可以在一个数组中存多种类型数据)

缺点:每个结构体都包含完整的信息,在很多时候是多余的。

而numpy中虽然数组类型是固定的(一般情况下只能是同类型数据),但是更高效。

学习了NumPy中的数组:

了解了数组的属性:数组大小(所有元素的总长度)、形状(如3*4),数据类型(int,float等),字符长度等

对数组进行索引、切片(与python基础语法基本相同)
\section{12月9日}
\subsection{python}
NumPy中,对多维数组进行切片,以获取行和列,此时的切片数组是原数组的视图,若对其进行修改,会使原数组相应位置也发生改变(非副本视图)。

数组的复制(numpy.copy)(副本)。

利用numpy.reshape将数组变形,如:单行或单列数组变成矩阵。

利用numpy中concatenate,vstack和hstack等进行数组的拼接。(其实用concatenate都可以实现,但后者更简洁):

 numpy.vstack([array1,array2])

 numpy.concatenate([array1,array2])

 \#以上两个等价

 numpy.hstack([array1,array2])

 numpy.concatenate([array1,array2],axis = 1)

 \#以上两个等价

以及数组的分裂(split),分裂是以输入的值为下标作为另一个新数组的首元素,

如:

 x = [1,2,3,4]

 x1,x2 = np.split(x,[2])

此时x1 = [1,2], x2 = [3,4]

以上的对数组操作可以混合使用,灵活使用以满足不同的需求。
\section{12月10日}
\subsection{python}
NumPy模块的学习:

python中一些循环耗时过长,而用向量的方式进行计算可以有效地提高代码运行效率。

了解了numpy中的通用函数(包括四则运算、绝对值、三角函数以及指数对数等)

对数组进行运算,实际上就是分别对数组中每一个元素进行运算,再返回一个新的(运算后的)数组。


利用out可以将运算后数组指定输出到想要储存的位置(包括数组切片的视图,从而修改数组的部分元素):

 x = numpy.arange(4)  \#创建一个数组

 y = numpy.empty(4)  \#再创建一个空的等长数组

\#y = numpy.empty(len(x)) 也可以这样


 np.multiply(x, 10, out=y) \#指定输出到y

 print(y) 

\#结果[ 0. 10. 20. 30. 40.]

聚合:用reduce(之前学过的高阶函数进行累积计算)返回一个最终计算结果。

外积:(outer)将两组数组所有数的两两组合进行运算,并把所有结果都返回。

如:

np.multiply.outer([1,2,3], [1,2,3])

会得到:

array([[1,2,3],

[2,4,6],

[3,6,9]])

过程实际上是:

array([[1*1,1*2,1*3],

[2*1,2*2,2*3],

[3*1,3*2,3*3]])
\section{12月11日}
\subsection{LaTex}
学习了LaTex的基础语法,了解了它的常用的控制序列的格式,可以掌握基本排版文章的水平。

基础排版格式:

$\backslash$ documentclass[UTF8]{ctexart} \%文档类别说明

$\backslash$ usepackage{amsmath} \%可以通过$\backslash$ usepackage引入宏包,类似import

$\backslash$ title{hello,world} \%标题

$\backslash$ author{Allan} \%作者名

$\backslash$ date{\today} \%日期($\backslash$ today可以生成当前日期)

$\backslash$ begin{document} \%与$\backslash$ end配对,中间为环境,此处环境为(document)

$\backslash$ maketitle \%生成导言区中的标题等

$\backslash$ tableofcontents \%(放在maketitle后)可生成目录

$\backslash$ section{这一级标题}

内容1

$\backslash$ subsection{这是二级标题}

内容2

$\backslash$ paragraph{段落}

内容3

$\backslash$ subparagraph{这是二级段落}

内容4

$\backslash$ subsubsection{这是三级标题}

内容5

$\backslash$ end{document}

上面通过$\backslash$ usepackage导入(amsmath)宏包以引入数学功能,从而进行数学公式的插入与使用。

具体的数学公式用法待使用时再去学习。

\end{document}